\documentclass{article}
\usepackage{graphicx} % Required for inserting images

\title{Advanced Time Series Analysis}
\author{Alexander Unger}
\date{December 2024}

\begin{document}

\maketitle

\section{Task2}

\subsection{Task C}
\subsection{Task A)}
By examining the realization of the process, the time series appears to oscillate consistently around a constant mean. The graph confirms this behavior, showing fluctuations around the expected value of 40, with no indication of a deterministic trend or changes in variance over time.
Given this observation, the time series exhibits a drift component, and the second case of the Dickey-Fuller test (with a constant drift) is the most appropriate.
$$y_t = c + \phi * y_{t-1} + \epsilon_t$$
where :\\
$c$ : can be seen as the constant drift term\\
$\phi y_{t-1}$ is the autoregressive component.\\
$\epsilon_t$ is the innovation component.\\

In our opinion \textbf{case one} can be rejected as the series does not oscillate around zero but around a non-zero mean, which here is the expected value $E[y_t]$=40
and the constant c=2 includes a drift which, invalidates the first case. \\
\textbf{Case four} can also be rejected, as the series does not exhibit a deterministic trend (e.g., a steady upward or downward slope).
The ARMA(1,1) model lacks a trend component $\beta$, making the fourth case unnecessary

\section{Task3}

\subsection{Task 3b}
For case 1 we get an likelihood value of -1259.2685. 
\begin{itemize}
    \item These parameters are closer to the data-generating process (Task 2.1), which had similar values.
    \item The relatively higher log likelihood (less negative) indicates a better fit to the observed data.
    \item The v=4 account for potential heavy tails in the data.
\end{itemize}
For case 2 we get a log-likelihood value of -5562.2279. These values deviate significantly from the data generating process.
\begin{itemize}
    \item c = 1.5 is lower than the original, shifting the mean of the process.
    \item $\phi$ = 0.75 suggests less persistence compared to
    \item$\theta$ = 0.5 introduces more weight to the moving average component.
    \item v = 6 reduces the heavy-tailed nature of the innovations, making the model less flexible for capturing outliers.
    \item The much lower log likelihood (more negative) indicates a poor fit to the observed data.
\end{itemize}
\textbf{Overall}: A higher (less negative) log likelihood value indicates better fit. Case a) aligns more closely with the underlying process and thus fits the data better than case b).
The large difference in logarithmic likelihoods (1259.2685 vs. 5562.2279) confirms that case a) is significantly better at capturing the underlying structure of the time series.
\end{document}
